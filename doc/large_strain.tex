%\documentclass[12pt]{article}
\documentclass[preprint,12pt,times]{elsarticle}

% \usepackage{url}

% \usepackage[linesnumbered,lined,commentsnumbered,ruled]{algorithm2e}
%\usepackage{amssymb}
\usepackage{amsmath}
%\usepackage{amsfonts}
\usepackage{bm}
%\usepackage{bbm}
%\usepackage{caption}
%\usepackage{color}
%\usepackage{geometry}
%\usepackage{graphicx}
%\usepackage[colorlinks,linkcolor=blue,citecolor=blue,urlcolor=blue]{hyperref}
%\usepackage{layouts}
%\usepackage{mathrsfs}
%\usepackage{pgfkeys,pgfmath,pgfcore}
%\usepackage{srcltx}
%\usepackage[labelformat=simple]{subcaption}
%\usepackage{xspace}

%\renewcommand{\thetable}{\arabic{table}}
%\captionsetup[table]{labelformat=simple, labelsep=colon}

%\pgfkeys{/pgf/number format/.cd,1000 sep={}}

%% see http://tex.stackexchange.com/questions/2441/how-to-add-a-forced-line-break-inside-a-table-cell
%\newcommand{\specialcell}[2][c]{%
%  \begin{tabular}[#1]{@{}c@{}}#2\end{tabular}}

% put your own definitions here:
% \renewcommand\thesubfigure{(\alph{subfigure})}
\def\gz  #1{           \mbox{$\boldsymbol{#1}$}}
% \DeclareMathAlphabet{\bsf}{OT1}{cmss}{bx}{n}
% \def\msf  #1{           \mbox{\!\!      $\sf #1$}}
% \def\Grad #1 {{\rm Grad} #1}
% \def\grad #1 {{\rm grad} #1}
\def\Div {\mbox{Div\,}}
\def\div {\mbox{div\,}}
\def\d {\,\mbox{d}}
\def\D {\,\mbox{D}}
% \newcommand{\norm}[1]{\left\lvert #1 \right\rvert}
%\newcommand\normDouble[1]{\left\lVert#1\right\rVert}
%\newcommand{\normXY}[1]{\lvert #1 \rvert_{2d}}
%\newcommand{\dbracket}[1]{\left[\!\!\left[ #1 \right]\!\!\right]}

\def\mcl  #1{               {\cal #1}}
%\def\bcl  #1{\mbox{\boldmath$\cal #1$}}


\begin{document}

\section{Theoretical background}

A brief summary of balance equations in large strain hyperelastic continuum.

\subsection{Kinematics}

Deformation of the body $\mcl B$ from the reference configuration $\mcl B_0$ to the current configuration $\mcl B_t$
is defined via the mapping $\gz x = \gz \varphi (\gz X,t)$.

Linear deformation map is assumed for the macroscopic point
\begin{align}
\d \gz x = \gz F \cdot \d \gz X,
\end{align}
where $\gz F := \gz \nabla_{X} \, \gz \varphi$ is the deformation gradient.

\subsection{Kinetics}

For conservative systems the total potential energy is introduced as
\begin{align}
\mcl E = \int_{\mcl B_0} \mcl U_0(\gz \varphi, \gz F; \gz X) \,
\end{align}
that must be stationary in order for the system to be in equilibrium:
\begin{align}
\delta \mcl E = 0 \, .
\label{eq:stationary}
\end{align}

\section{Finite Element discretization}

We now introduce a FE triangulation $\mathcal{T}^h$ of $\mcl B_0$ and
the associated FE space of continuous piecewise elements of a fixed polynomial degree. % : $V^h \subset H^1 (\mcl B_0)$.
Macro- -deformation map are given in a
a vector space spanned by standard vector--valued FE basis functions $\gz N^i(\gz x)$ (e.g. polynomials with local support):
\begin{alignat}{2}
       \gz \varphi^h &=:  \sum_{i \in \mcl I_\varphi}       \varphi_i \gz N^i_\varphi (\gz X) \quad \quad \quad
\delta \gz \varphi^h &&=: \sum_{i \in \mcl I_\varphi} \delta \varphi_i \gz N^i_\varphi (\gz X) \,,
\end{alignat}
where superscript $h$ denotes that this representation is related to the FE mesh with size function $h(\gz X)$ and $\mcl I_\varphi$ is the sets of unknown degrees of freedom for the
macroscopic fields. For the sake of this study $\gz N^i_\varphi (\gz X)$ are such that only one component of the vector shape function is non-zero for each $i$.

\bibliographystyle{elsarticle-num}
\bibliography{bibliography}

\end{document}
