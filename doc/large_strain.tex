%\documentclass[12pt]{article}
\documentclass[preprint,12pt,times]{elsarticle}

% \usepackage{url}

% \usepackage[linesnumbered,lined,commentsnumbered,ruled]{algorithm2e}
%\usepackage{amssymb}
\usepackage{amsmath}
%\usepackage{amsfonts}
\usepackage{bm}
%\usepackage{bbm}
%\usepackage{caption}
%\usepackage{color}
%\usepackage{geometry}
%\usepackage{graphicx}
%\usepackage[colorlinks,linkcolor=blue,citecolor=blue,urlcolor=blue]{hyperref}
%\usepackage{layouts}
%\usepackage{mathrsfs}
%\usepackage{pgfkeys,pgfmath,pgfcore}
%\usepackage{srcltx}
%\usepackage[labelformat=simple]{subcaption}
%\usepackage{xspace}

\usepackage[linesnumbered,lined,commentsnumbered,ruled]{algorithm2e}

%\renewcommand{\thetable}{\arabic{table}}
%\captionsetup[table]{labelformat=simple, labelsep=colon}

%\pgfkeys{/pgf/number format/.cd,1000 sep={}}

%% see http://tex.stackexchange.com/questions/2441/how-to-add-a-forced-line-break-inside-a-table-cell
%\newcommand{\specialcell}[2][c]{%
%  \begin{tabular}[#1]{@{}c@{}}#2\end{tabular}}

% put your own definitions here:
% \renewcommand\thesubfigure{(\alph{subfigure})}
\def\gz  #1{           \mbox{$\boldsymbol{#1}$}}
% \DeclareMathAlphabet{\bsf}{OT1}{cmss}{bx}{n}
% \def\msf  #1{           \mbox{\!\!      $\sf #1$}}
% \def\Grad #1 {{\rm Grad} #1}
% \def\grad #1 {{\rm grad} #1}
\def\Div {\mbox{Div\,}}
\def\div {\mbox{div\,}}
\def\d {\,\mbox{d}}
\def\D {\,\mbox{D}}
% \newcommand{\norm}[1]{\left\lvert #1 \right\rvert}
%\newcommand\normDouble[1]{\left\lVert#1\right\rVert}
%\newcommand{\normXY}[1]{\lvert #1 \rvert_{2d}}
%\newcommand{\dbracket}[1]{\left[\!\!\left[ #1 \right]\!\!\right]}

\def\mcl  #1{               {\cal #1}}
%\def\bcl  #1{\mbox{\boldmath$\cal #1$}}

%\DeclareMathOperator{\det}{det}
\DeclareMathOperator{\trace}{tr}
\newcommand{\diff}{\mathop{}\!\mathrm{d}}


\begin{document}

\section{Introduction}

TODO: elasticity and no matrix-free. fluids used a lot.

\section{Theoretical background}

A brief summary of balance equations in large strain hyperelastic continuum.

\subsection{Kinematics}

Deformation of the body $\mcl B$ from the reference configuration $\mcl B_0$ to the current configuration $\mcl B_t$
is defined via the mapping $\gz x = \gz \varphi (\gz X,t)$.

Linear deformation map is assumed for the macroscopic point
\begin{align}
\d \gz x = \gz F \cdot \d \gz X,
\end{align}
where $\gz F := \gz \nabla_{X} \, \gz \varphi$ is the deformation gradient.

\subsection{Kinetics}

For conservative systems the total potential energy is introduced as
\begin{align}
\mcl E = \int_{\mcl B_0} \mcl U_0(\gz \varphi, \gz F; \gz X) \,
\end{align}
that must be stationary in order for the system to be in equilibrium:
\begin{align}
\delta \mcl E = 0 \, .
\label{eq:stationary}
\end{align}

\subsection{Linearization}

TODO: linearize and show how tangent look like.
Also show what it boils down in the case of small strain.

\subsection{Constitutive modelling}

Neo-Hookean model
\begin{gather}
\psi \left( \mathbf{C} \right)
  = \frac{\mu}{2} \left[ \trace{\mathbf{C}} - \trace{\mathbf{I}} - 2 \ln\left( J \right) \right]
  + \lambda \ln^{2}\left( J \right)
\end{gather}
where $\mu$ and $\lambda$ respectively denote the shear modulus and Lam\'{e} parameter,
and the volumetric Jacobian $J = \det\left(\mathbf{F}\right) = \sqrt{\det\left(\mathbf{C}\right)}$.
\begin{gather}
\frac{d \psi \left( \mathbf{C} \right)}{d \mathbf{C}}
  = \frac{\mu}{2} \mathbf{I} - \frac{1}{2} \left[ \mu - 2\lambda\ln\left( J \right) \right] \mathbf{C}^{-1}
\end{gather}
\begin{gather}
\frac{d^{2} \psi \left( \mathbf{C} \right)}{d \mathbf{C} \otimes d \mathbf{C}}
  = \frac{1}{2}\left[ \mu - 2\lambda\ln\left( J \right) \right] \left[ - \frac{d \mathbf{C}^{-1}}{d \mathbf{C}} \right]
  + \frac{\lambda}{2} \mathbf{C}^{-1} \otimes \mathbf{C}^{-1}
\end{gather}

Denoting $\chi\left( \bullet \right)$ as the Piola transformation, which implies
\begin{gather}
\chi\left( \mathbf{A} \right)_{ij}
  = F_{iA} A_{AB} F_{jB} \\
\chi\left( \boldsymbol{\mathcal{A}} \right)_{ijkl}
  = F_{iA} F_{jB} \mathcal{A}_{ABCD} F_{kC} F_{lD}
\end{gather}
for rank-2 and rank-4 tensors $\mathbf{A}$ and $\boldsymbol{\mathcal{A}}$, then the Kirchhoff stress and its associated material tangent are
\begin{gather}
\boldsymbol{\tau}
  \equiv J \boldsymbol{\sigma}
  = \chi\left( 2 \frac{d \psi \left( \mathbf{C} \right)}{d \mathbf{C}} \right)
  = \mu \mathbf{b} - \left[ \mu - 2\lambda\ln\left( J \right) \right] \mathbf{I}
\end{gather}
and
\begin{gather}
J \boldsymbol{\mathcal{C}}
  = \chi\left( 4 \frac{d^{2} \psi \left( \mathbf{C} \right)}{d \mathbf{C} \otimes d \mathbf{C}} \right)
  = 2 \left[ \mu - 2\lambda\ln\left( J \right) \right] \boldsymbol{\mathcal{S}}
  + 2 \lambda \mathbf{I} \otimes \mathbf{I}
\end{gather}
where $\boldsymbol{\mathcal{S}}$ is the fourth-order symmetric identity tensor.
The action that $J \boldsymbol{\mathcal{C}}$ performs when contracted with an arbitrary rank-2 symmetric tensor is therefore
\begin{gather}
J \boldsymbol{\mathcal{C}} : \left( \bullet \right)
  = 2 \left[ \mu - 2\lambda\ln\left( J \right) \right] \left( \bullet \right)
  + 2 \lambda \trace\left( \bullet \right) \mathbf{I}
\end{gather}

\section{Finite Element discretization}

We now introduce a FE triangulation $\mathcal{T}^h$ of $\mcl B_0$ and
the associated FE space of continuous piecewise elements of a fixed polynomial degree. % : $V^h \subset H^1 (\mcl B_0)$.
Macro- -deformation map are given in a
a vector space spanned by standard vector--valued FE basis functions $\gz N^i(\gz x)$ (e.g. polynomials with local support):
\begin{alignat}{2}
       \gz \varphi^h &=:  \sum_{i \in \mcl I_\varphi}       \varphi_i \gz N^i_\varphi (\gz X) \quad \quad \quad
\delta \gz \varphi^h &&=: \sum_{i \in \mcl I_\varphi} \delta \varphi_i \gz N^i_\varphi (\gz X) \,,
\end{alignat}
where superscript $h$ denotes that this representation is related to the FE mesh with size function $h(\gz X)$ and $\mcl I_\varphi$ is the sets of unknown degrees of freedom for the
macroscopic fields. For the sake of this study $\gz N^i_\varphi (\gz X)$ are such that only one component of the vector shape function is non-zero for each $i$.

\section{Matrix-free operator evaluation}

Due to the specifics of matrix-free operator evaluation in \texttt{deal.II}, when performing contraction with gradient of shape functions in current (deformed) configuration, the integration is also done over the deformed configuration. Therefore in order to arrive at the integral in undeformed configuration, additionally we have to divide by Jacobian $J$ of the mapping at a given quadrature point.

\begin{algorithm}[h]
  \SetKwInOut{Input}{Given}
  \SetKwInOut{Output}{Return}
  \Input{RHS FE vector $\gz x$, current FE solution $\gz u$,
  cacheed $c_1 := \mu - 2 \lambda \log(J)$ for each cell and quadrature point}
  \Output{action of the FE tangent on $\gz x$}
  \ForEach{ element $K \in \Omega^h$ }{
        \ForEach{quadrature point $q$ on $K$}{
          evaluate $\gz \nabla_X \gz u^h$ \tcp*{2nd order}
          evaluate $\gz F = \gz I + \gz \nabla_X \gz u^h$ \tcp*{2nd order}
          evaluate $J = \rm{det}(\gz F)$ \tcp*{scalar}
          evaluate $\gz b = \gz F \cdot \gz F^T$ \tcp*{2nd order symmetric}
          evaluate $\gz x_1 := \gz \nabla \gz x$ \tcp*{2nd order}
          evaluate $\gz \tau = \mu \gz b - c_1 \gz I$ \tcp*{2nd order symmetric}
          evaluate $J \boldsymbol{\mathcal{G}} = x_1 \cdot \gz \tau$ \tcp*{2nd order symmetric}
          ``submit'' contraction $\left[\gz \nabla \gz N_i : J\boldsymbol{\mathcal{G}}\right] / J$\;
          evaluate $\gz x_2 := \gz \nabla_{sym} \gz x$ \tcp*{2nd order symmetric}
          evaluate $J \boldsymbol{\mathcal{C}} = 2 c_1 \gz x_2 + 2 \lambda \rm{tr}( \gz x_2) \gz I $ \tcp*{2nd order symmetric}
          ``submit'' contraction $\left[\gz \nabla_{sym} \gz N_i : J \boldsymbol{\mathcal{C}} \right]/ J$ \;
        }
  }
  \caption{Matrix-free tangent cell operator, variant 1: minimum caching.}
  \label{alg:mf_multivector}
\end{algorithm}

TODO[DD]: summary of what's happening, how operator is evaluated, what we do, what operations (determinant, etc), count number of FLOPS

\section{Geometric multigrid preconditioning}

TODO[DD]: explain how this works with GMG (eulerian mapping on fine level gets restricted onto the coarse levels, the rest works as-is).
Hopefully we can still use some mappings on coarse level to better represent geometric of inclusions (pores).

\section{Numerical examples}

TODO: question to asnwer -- how much to cache? At least $detF ^ {2/3}$. Other options is stress and strain at quadrature points.
Run benchmarks and see.

\subsection{Small strain linear elasticity}

TODO: summarize Mathias results (head and Cook's membrane)

TODO: switch head to quasi-static load instead of dynamics?

\subsection{Large strain Neo-Hook elasticity}

TODO: add new results from Cook's and some matrix-inclusion (matrix-pore).
Eliptical holes are easier as we don't need heterogeneous operator.

\section{Summary and Conclusions}

TODO: summarize results, good/bad, where can be applied,...
Bio-med fluid-structure ?

\bibliographystyle{elsarticle-num}
\bibliography{bibliography}

\end{document}
